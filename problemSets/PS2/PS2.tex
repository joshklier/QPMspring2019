\documentclass[12pt,letterpaper]{article}
\usepackage{graphicx,textcomp}
\usepackage{natbib}
\usepackage{setspace}
\usepackage{fullpage}
\usepackage{color}
\usepackage[reqno]{amsmath}
\usepackage{amsthm}
\usepackage{fancyvrb}
\usepackage{amssymb,enumerate}
\usepackage[all]{xy}
\usepackage{endnotes}
\usepackage{lscape}
\newtheorem{com}{Comment}
\usepackage{float}
\usepackage{hyperref}
\newtheorem{lem} {Lemma}
\newtheorem{prop}{Proposition}
\newtheorem{thm}{Theorem}
\newtheorem{defn}{Definition}
\newtheorem{cor}{Corollary}
\newtheorem{obs}{Observation}
\usepackage[compact]{titlesec}
\usepackage{dcolumn}
\usepackage{tikz}
\usetikzlibrary{arrows}
\usepackage{multirow}
\usepackage{xcolor}
\newcolumntype{.}{D{.}{.}{-1}}
\newcolumntype{d}[1]{D{.}{.}{#1}}
\definecolor{light-gray}{gray}{0.65}
\usepackage{url}
\usepackage{listings}
\usepackage{color}

\definecolor{codegreen}{rgb}{0,0.6,0}
\definecolor{codegray}{rgb}{0.5,0.5,0.5}
\definecolor{codepurple}{rgb}{0.58,0,0.82}
\definecolor{backcolour}{rgb}{0.95,0.95,0.92}

\lstdefinestyle{mystyle}{
	backgroundcolor=\color{backcolour},   
	commentstyle=\color{codegreen},
	keywordstyle=\color{magenta},
	numberstyle=\tiny\color{codegray},
	stringstyle=\color{codepurple},
	basicstyle=\footnotesize,
	breakatwhitespace=false,         
	breaklines=true,                 
	captionpos=b,                    
	keepspaces=true,                 
	numbers=left,                    
	numbersep=5pt,                  
	showspaces=false,                
	showstringspaces=false,
	showtabs=false,                  
	tabsize=2
}
\lstset{style=mystyle}
\newcommand{\Sref}[1]{Section~\ref{#1}}
\newtheorem{hyp}{Hypothesis}

\title{Problem Set 2}
\date{Due: February 27, 2018}
\author{Quantitative Political Methodology (U25 363)}

\begin{document}
	\maketitle

	
	\section*{Instructions}
	\begin{itemize}
		\item Please show your work if possible. You may lose points by simply writing in the answer. If the problem requires you to execute commands in \texttt{R}, please include the code you used to get your answers. Please also include the \texttt{.R} file that contains your code. If you have plots, attach them as well within your written document. Make sure you label clearly which question the codes correspond to. If you are not sure if work needs to be shown for a particular problem, please ask me.
		\item Your homework should be submitted electronically on the course GitHub page.
		\item This problem set is due before the beginning of class on Wednesday February 27, 2019. No late assignments will be accepted.
		\item Total available points for this homework is 100.
	\end{itemize}

\section*{Question 1 (5 points)}
You would like to find the proportion of bills passed by Congress that were vetoed by the
President in the last congressional session. After checking congressional records, you see that
for the population of all 40 bills passed, 2 were vetoed. Does it make sense to construct a
confidence interval using these data to answer your question? Explain.

No it doesnt make sense to construct a confidence interval with this data as the sample size, n, should be sufficiently large that there are at least 15 observations for both sides of the argument. In this problem there are 38 observations where there was no veto, but only 2 that were vetoed.

%What proportion of a normal distribution falls in the following ranges (round to 3 decimal points)? Use Table A on the last page of Agresti and Finlay's book. Highlight the appropriate portion of the graph. \\
%\vspace{0.25in}
%
%\begin{itemize}
%	\item[(a)] Above a $z$-score of 1.96.
%
%\begin{figure}[H]
%\centering
%\includegraphics[trim=0.1cm 0.1cm 0cm 0.1cm, clip=true, width=0.4\textwidth]{normal_dist.jpg}
%\end{figure}
%
%\vspace{0.8cm}
%
%
%\item[(b)] Below a $z$-score of 1.96.  
%
%\begin{figure}[H]
%\centering
%\includegraphics[trim=0.1cm 0.1cm 0cm 0.1cm, clip=true, width=0.4\textwidth]{normal_dist.jpg}
%\end{figure}
%
%\vspace{0.8cm}
%
%
%\item[(c)] Above a $z$-score of $-$1.96.  
%
%\begin{figure}[H]
%\centering
%\includegraphics[trim=0.1cm 0.1cm 0cm 0.1cm, clip=true, width=0.4\textwidth]{normal_dist.jpg}
%\end{figure}
%
%\vspace{0.8cm}
%
%
%\item[(d)] Between the $z$-score of $-$1.96 and 1.96.  
%
%\begin{figure}[H]
%\centering
%\includegraphics[trim=0.1cm 0.1cm 0cm 0.1cm, clip=true, width=0.4\textwidth]{normal_dist.jpg}
%\end{figure}
%
%
%\end{itemize}

\section*{Question 2 (25 points)}The distribution of family size in a particular tribal society is skewed to the right, with $\mu=5.2$ and $\sigma=3$. Those values are unknown to an anthropologist, who samples families to estimate mean family size. For a random sample of 36 families, she gets a mean of 4.6 and a standard deviation of 3.2.

\begin{itemize}

\item[(a)] Identify the population distribution. State its mean and standard deviation. Is the data skewed?\\
mean for population= 5.2
stdev for population= 3
skewed to the right 

\item[(b)] Identify the sample data distribution. State its mean and standard deviation. Is the data skewed?\\

mean=4.6
stdev=3.2
the data is skewed as it doesn't match the population parameter.
\item[(c)] Identify the sampling distribution of $\bar{y}$. State its mean and standard error and explain what it describes.\\
mean=4.6 stdev=3.2
this is the sampling distribution of the sample mean. Meaning, we are looking at the distribution of the sample mean derived from a sample (n=36) of the original distribution.

\item[(d)] Find the probability that her sample mean falls within 0.5 of the population mean.\\
pnorm(5.7, mean=4.6, sd=3.2, lower.tail=TRUE)-pnorm(4.7, mean=4.6, sd=3.2, lower.tail=TRUE)
.122
\item[(e)] Suppose she takes a random sample of size 100. Find the probability that the sample mean falls within 0.5 of the true mean, and compare the answer to that in (d).\\

randomsample <-rnorm(100, mean=5.7, sd=3.2)
pnorm(randomsample, mean=5.7, sd=3.2)-pnorm(randomsample, mean=4.7, sd=3.2)

\item[(f)] Refer to (e). If the sample were truly random, would you be surprised if the anthropologist obtained $\bar{y}=4$. Why?\\
no because it is within the standard deviation. 
\end{itemize}

\section*{Question 3 (10 points)}
The GSS asks respondents to rate their political views on a seven-point scale, where 1= extremely liberal, 4=moderate, and 7=extremely conservative. A researcher analyzing data from 2011 has the following data

\begin{table}[H]
\centering
\begin{tabular}{cccccc}
\hline \\[-1.8ex]
Variable & $N$ & Mean  & St. Dev & SE Mean \\ 
\hline \\[-1.8ex]
\texttt{Polviews} & 1294 & 4.23  & 1.39 & 0.0387 \\ 
\hline
\end{tabular} 
\end{table}

\begin{itemize}

\item[(a)] Show how to construct a 95\% confidence interval from the information provided.\\
n=1294, z=1.96, mean=4.23, sd=1.39
4.23+1.96(1.39/sqrt(1294)) = 4.23+.0757
4.23-1.96(1.39/sqrt(1294)) = 4.23-.0757
\item[(b)] Interpret the confidence interval you found in (a).\\
This means that we can be 95 percent confident that the population mean falls between the values of 4.1543 and 4.3057

\item[(c)] Would the confidence interval be wider or narrower (i) if you constructed a 90\% confidence interval, (ii) if you found the 95\% confidence interval only for those who called themselves \textit{strong Democrats} on political party identification (PARTYID), for whom the mean was 3.50 with standard deviation 1.61?\\
(i) the confidence interval would be narrower with a 90 percent confidence interval. This is because as the confidence level decreases (95 percent to 90 percent), the width of the confidence level decreases.
(ii) The width would increase because we are only sampling the "strong democrats" which means that our n is decreasing. Since n is in the denominator, as it decreases, the width increases.
\end{itemize}

\section*{Question 4 (5 points)}
For a normal distribution with $\mu = 50$ and $\sigma^2 = 36$, find the probability that an observation falls (Hint: type \texttt{help(Normal)} in \texttt{R}):

\begin{itemize}
\item[(a)] At or below the value 57.75 \\
pnorm(57.75, mean=50, sd=6)
.901
\item[(b)] At or above the value of  50.45 \\
pnorm(50.45, mean=50, sd=6, lower.tail=FALSE)
.470
\item[(c)] Between the values of 52.4 and 59.4 \\
pnorm(59.4, mean=50, sd=6, lower.tail=TRUE)-pnorm(52.4, mean=50, sd=6, lower.tail=TRUE)
.285

\end{itemize}
\section*{Question 5 (5 points)}

\texttt{R} has a number of functions that make it simple to simulate from a variety of distributions.\\

\noindent One thing to note is that when sampling you want to set a seed in \texttt{R}.  Setting the seed allows you to replicate your results.  It doesn't matter what it is set to.  So, for the purposes of this question, type:\\

\noindent \texttt{set.seed(12345)}\\

\noindent Suppose that salaries follow a normal distribution with mean 40000 and standard deviation 15000.  We can sample from this distribution using the \texttt{rnorm()} command.  Type the following into \texttt{R} to generate a sample with 10000 observations: \\

\noindent \texttt{salaries <- rnorm(n=10000,mean=40000,sd=15000)}\\

\noindent Plot the distribution.  Add a title to this plot.  Save this plot as a .pdf file.  \\
set.seed(12345)
salaries <- rnorm(n=10000, mean=40000, sd=15000)
density(salaries)
plot(density(salaries), xlab="salary amount", main="Denisty plot of salaries")


\section*{Question 6 (10 points)}
Plot probability density functions for the following normal distributions. Make all the plots on a single page. Make sure your plots have properly labeled titles and axes, and your axes are comparable across plots. \\

\begin{itemize}
\item[(a)]  Normal Distribution with $\mu=0$ and $\sigma^2=0.4$

x<- seq(-10, 10, length=100)
normal.dist<- dnorm(x,0,sqrt(.4))
plot(normal.dist,type="l", lty=1)
title("plot 6a")
\item[(b)]  Normal Distribution with $\mu=0$ and $\sigma^2=3$

normal.dist2<- dnorm(x,0,sqrt(3))
plot(normal.dist2,type="l", lty=1)
title("plot 6b")
\item[(c)]  Normal Distribution with $\mu=3$ and $\sigma^2=3$

normal.dist3<- dnorm(x,3,sqrt(3))
plot(normal.dist3,type="l", lty=1)
title("plot 6c")
\item[(d)]  Normal Distribution with $\mu=3$ and $\sigma^2=0.4$

normal.dist4<- dnorm(x,3,sqrt(.4))
plot(normal.dist4,type="l", lty=1)
title("plot 6d")
\item[(e)]  Normal Distribution with $\mu=-2$ and $\sigma^2=5$

normal.dist5<- dnorm(x,-2,sqrt(5))
plot(normal.dist5,type="l",lty=1)
title("plot 6e")
\item[(f)] Normal Distribution with $\mu=-2$ and $\sigma^2=\frac{1}{4}$

normal.dist6<- dnorm(x,-2,sqrt(.25))
plot(normal.dist6,type="l",lty=1)
title("plot 6f")

\end{itemize}

\section*{Question 7 (20 points)}
Peake and Eshbaugh-Soha (2008) study drug policy coverage. Their data count the number of nightly television news stories in a month focusing on drugs, from January 1977 to December 1992. The dataset is in comma-separated format in the file named drugCoverage.csv. Download it from \href{https://dataverse.harvard.edu/dataset.xhtml?persistentId=doi:10.7910/DVN/ARKOTI}{Monogan (2015)'s Dataverse}. The variables in the dataset are: a character-based time index showing month and year (\texttt{Year}), news coverage of drugs (\texttt{drugsmedia}), an indicator for a speech on drugs that Ronald Reagan gave in September 1986 (\texttt{rwr86}), an indicator for a speech George H.W. Bush gave in September 1989 (\texttt{ghwb89}), the president's approval rating (\texttt{approval}), and the unemployment rate (\texttt{unemploy}).

\begin{itemize}

\item[(a)]  Draw a histogram of the monthly count of drug-related stories. \\
 hist(drugcoverage$drugsmedia, xlab="months", ylab="frequency", main = "monthly count of drug-related charges")

\item[(b)] Draw two boxplots: One of drug-related stories and another of presidential approval. How do these figures differ and what does that tell you about the contrast between the variables? \\
boxplot(drugcoverage$drugsmedia, xlab="drugs in media", ylab="frequency" ,main="boxplot of drugs in media")
boxplot(drugcoverage$approval, ylab="frequency", xlab="approval rating", main="presidental approval")

The approval rating is a lot higher than the amount of drug-related stories. Also, the variability in approval rating is a lot higher than the variability in drug-related stories.
\item[(c)]  Draw two scatterplots:
\begin{itemize}
\item In the first, represent the number of drug-related stories on the vertical axis, and place the unemployment rate on the horizontal axis.

plot(x=drugcoverage$unemploy, y=drugcoverage$drugsmedia, main = "First scatterplot", xlab = "unemployment", ylab= "drug-related stories")

\item In the second, represent the number of drug-related stories on the vertical axis, and place presidential approval on the horizontal axis.

plot(x=drugcoverage$approval, y=drugcoverage$drugsmedia, main = "second scatterplot", xlab="approval rating", ylab="drug-related stories")

\item How do the graphs differ? What do they tell you about the data? \\
\end{itemize}

As approval rating increases, the number of drug-related stories also increases.
As unemployment decreases, the number of drug-related stories increases.
This tells us that the number of drug-related stories differs based on both the approval rating and the unemployment rate.

\item[(d)]  Draw two line graphs:
\begin{itemize}
\item In the first, draw the number of drug-related stories by month over time.

plot(x=drugcoverage$Year, y=drugcoverage$drugsmedia, ylab="drug related stories", xlab="month", main="number of drug-related stories over time")

\item In the second, draw presidential approval by month over time.

plot(x=drugcoverage$Year, y=drugcoverage$approval, ylab="approval rating", xlab="month", main="approval rating over time")

\item What can you learn from these graphs? \\
the approval rating is varys a lot over the course of a presidents tenure. While the number of drug related stories has some outliers but mainly stays below 50 stories per month.

\end{itemize}



\end{itemize}

\section*{Question 8 (20 points)}
For this question, you will work with W-NOMINATE data to trace the policy positions of members in the \href{https://drive.google.com/open?id=1kVn-Asz8KXPQhiLSQGEabdVJh9_8_wjf}{U.S. House of Representatives}. With the data, you will learn about polarization (i.e. distance between the ideological positions of the Democratic Party and the Republican Party). You will also learn about the ideological cohesiveness of each party. Answer the following questions:\\

\begin{itemize}

\item[(a)] Import data on the 88th and 107th Congresses. Then, create four subsets of the data by session and party (Democratic Party in the 88th session, Democratic Party in the 107th session, Republican Party in the 88th session, and Republican Party in the 107th session). \\
read.csv("wnominatehouse.csv")
wnominate <- read.csv("wnominatehouse.csv")

> democrats107 <- subset(wnominate,congress==107 & party==100)
> democrats88 <- subset(wnominate, congress==88 & party==100)
> republicans88 <- subset(wnominate, congress==88 & party==200)
> republicans107 <- subset(wnominate, congress==107 & party==200)

\item[(b)] For the Democratic Party, calculate the median W-NOMINATE scores for two Congresses. How did the median change over time? What does this mean? \\
median(democrats88$wnominate, na.rm=TRUE)
median(democrats107$wnominate, na.rm=TRUE)
The median increased (away from 0), meaning that the middle democrat became more polarized.
\item[(c)] For the Republican Party, calculate the median W-NOMINATE scores for the two Congresses. How did the median change over time? What does this mean? \\
median(republicans88$wnominate, na.rm=TRUE)
median(republicans107$wnominate, na.rm=TRUE)
the median increased, meaning that the midde republican became more polarized.
\item[(d)] 
For the Democratic Party, calculate the standard deviation of W-NOMINATE scores for the two Congresses. How did the standard deviation change over time? What does this mean? \\
sd(democrats88$wnominate)
sd(democrats107$wnominate)
The stdev decreased from 88 to 107. This means that democrats began voting more similarily. 
\item[(e)] For the Republican Party, calculate the standard deviation of W-NOMINATE scores for the two Congresses. How did the standard deviation change over time? What does this mean? \\
sd(republicans88$wnominate)
sd(republicans107$wnominate)
The stdev decreased from 88 to 107. This means that republicans began voting more similarly. 
%\part[2] 
%For the 71st Congress, create a plot that overlays two histograms. One histogram should plot the distribution of W-NOMINATE scores for the Democratic Party. The other histogram should plot the distribution of W-NOMINATE scores for the Republican Party. 
%\vspace{2cm}

\item[(f)] For the 88th Congress, create a plot that overlays two histograms. One histogram should plot the distribution of W-NOMINATE scores for the Democratic Party. The other histogram should plot the distribution of W-NOMINATE scores for the Republican Party. (Hint: to overlay two histograms, you can run two separate \texttt{hist} commands but include an \texttt{add} argument in the second \texttt{hist} one.) \\

hist(republicans88$wnominate, xlab = "wnominate", main="88th congress histogram", col="red", xlim=c(-1,1))
hist(democrats88$wnominate, xlab="wnominate", main="88th congress histogram", col="blue", add=T)

\item[(g)]  
For the 107th Congress, create a plot that overlays two histograms. One histogram should plot the distribution of W-NOMINATE scores for the Democratic Party. The other histogram should plot the distribution of W-NOMINATE scores for the Republican Party. \\
hist(republicans107$wnominate, xlab="wnominate", main="107th congress histogram", col="red",xlim=c(-1,1))
 hist(democrats107$wnominate, xlab="wnominate", main="107th congress histogram", col="blue", add=T)

\item[(h)] 
Based on what you have done so far, compare the 88th Congress and the 107th Congress.
The 88th congress was much more bipartisan. There is a huge polarization of the parties between the 88th and 107th congress.


\begin{itemize}
\item Did polarization decrease, increase, or stay the same? Are both parties responsible for this or is one party responsible? \\
polarization increased. Both parties responsible.


\item For each party, what happened to the ideological cohesiveness of its members? Did it decrease, increase, or stay the same?\\
I'm a bit confused on the question, however, I'll try to answer it as best as I can. In the 88th congress some democrats voted republican on issues and vice versa. In the 107th congress we barely see that.


\end{itemize}



\end{itemize}



\end{document}

